\documentclass[11pt]{article}

 \renewcommand{\thefootnote}{}
\renewcommand{\baselinestretch}{1.2}
  \renewcommand{\arraystretch}{1.0}
  \voffset= -15mm \hoffset= -16mm \textheight 22 cm \textwidth 165 mm
\parskip 0mm

\usepackage{cite}
 \usepackage{amsmath}
%\usepackage{amscd}
\usepackage{amssymb}
\usepackage{graphicx}
\usepackage{ulem}
 %\usepackage{indentfirst}
%\usepackage{ifpdf}
%\usepackage{fancyhdr}
  \usepackage{hyperref}


 \begin{document}
 \title{OnionPhone V0.1a:\\ VOIP add-on for TorChat}

 \author{Van Gegel$^{1}$}
 \footnotetext{$^1$\,\textsf{torfone@ukr.net}}
   \footnotetext{ $^2$\,\textsf{TorFone and OnionPhone produced independently from the Tor� anonymity software and carries no guarantee from The Tor Project.} }
   \footnotetext{ $^3$\,\textsf{Alpha version is presented ``as is" and has not yet been checked by anyone except the developer. Do not use in ``life or death" cases.} }

\date{}
\maketitle
\begin{quotation}
\textbf{The idea.} OnionPhone was inspired by the previous TorFone \cite{Gegel12}  presented in 2012 - the first out-of-box solution for VOIP over Tor network suitable for practical use.  Later homegrown experiment grew into a small project which is due to its relevance now is cited in publicists \cite{Meredith13} and academic \cite{Rizal14, KFK14, Kambourakis14, SLL14}  works.

Despite the uniqueness of the TorFone has been criticized for mono-platform design (only Windows), the use of C ++ (hard-to-trace language-specific bugs are possible), the interweaving of the functional code and GUI and for deprecated cryptographic solutions from  Philip Zimmermann's PGPFone\cite{Zimmermann99}  (1995) on which the TorFone was fully based.

Therefore given the drawbacks was developed OnionPhone \cite{Gegel14} inherited the basic functionality of TorFone. Thanks Roger Dingledine a name of the project now is more appropriate to it's nature$^{2}$.

Since OnionPhone uses advanced cryptographic primitives and protocols it is not compatible with TorFone or any other VOIP solutions. Choosing between compatibility and efficiency we prefer the last and provided strongly optimization under Onion conditions. New project was designed in mind for high latency Onion transport and can be effectively used to further explore the possibility of VOIP over Tor$^{3}$.

\textbf{To Do:} the android version of OnionPhone and simple cross-platform Qt GUI connected over control port.
\end{quotation}
\pagebreak
\texttt{\section{General description}}
OnionPhone is a VOIP tool for calling over Tor network and can be used as a VOIP plugin for TorChat\cite{Kreuss12} . Call is  targeted to the Onion address of the recipient (it's hidden service HS). The recipient can install a reverse Onion connection to the originator's HS and use a faster channel periodically resetting the slower channel to reduce overall latency.

Also  provides the ability to switch to a direct UDP connection (with NAT traversal) after the connection is established over Tor (Tor instead SIP without one specific server, the registration and collection of metadata). And also OnionPhone can establish a direct UDP or TCP connection to the specified port on IP-address or host.

OnionPhone uses a proprietary protocol (not RTP)  with only one byte unencrypted header that  can provide some obfuscation of the DPI.

OnionPhone provides independent level of p2p encryption and authentication that is uses modern cryptographic primitives: Diffie-Hellmann key exchange on \texttt{Elliptic Curve 25519} and \texttt{Keccak Sponge Duplexing} encryption. In the case of a call to the Onion address Tor protects against MitM attacks. Also the recipient can verify identity of  the originator's Onion address (only with the permission of the sender) similar the TorChat authentication.  Otherwise possible multifactors authentication:
\begin{itemize}
\item voice (biometrics);
\item using previously shared password (with the possibility of hidden notification of coercion);
\item using a long-term public keys signed by PGP.
\end{itemize}
Onion Phone provides \texttt{Perfect Forward Secrecy} (uses a fresh key for each call) and \linebreak \texttt{Full  Deniability} for using long-term public keys of participants (as a fact and a content of the conversation): like a deniable SKEME \cite{RGK06} protocol uses for initial key exchange.

 OnionPhone can use a wide range of in-build voice codec (C source includes) from ultra low-bitrates up to high quality.  The full list is:
 MELPE-1200, MELP-2400, CODEC2-1300, CODEC2-3200, LPC10-2400, CELP4800, AMR-4750/12200+DTX,
 LPC-5600+VOCODER, G723-6400, G729-8000, GSM-HR-5600, GSM-FR-13200, GSM-EFR-12400, ILBC-13333, BV16-16000, OPUS-6000VBR, SILK-10000VBR, SPEEX-15200VBR+R.
Some of them are free, some require a license (check regional laws).
Implemented noise suppressors (NPP7 and SPEEX) of environment sounds and automatic mike gain control. Built-in LPC vocoder \cite{BZ11}  with the possibility of irreversible change of voice (robot, breathy etc.). Specially designed dynamic adaptive buffer useful for smart compensate of jitter in high latency Onion environment. Available the radio mode (Push to Talk)  and voice control (Voice Active Detector)  with generation a short signal when transmission is completed.

The OnionPhone source (available on $github$:\cite{Gegel141}):
\begin{itemize}
\item is a console style, does not require installation and can be run from removable disk or TrueCrypt container;
\item is fully open source, developed on pure C at the possible lowest  level and carefully commented;
\item statically linked, does not require additional third-party libraries and uses a minimum of system functions;
\item can be compiled under Linux OS (Debian, Ubuntu etc.) using GCC or under Win32 OS (from Windows 98 up to 8) using MinGW.
\end{itemize}


\section{Quick start and easy usage}
\subsection{Installation}
The easiest way to use OnionOhine as a VOIP plugin for TorChat:
\begin{itemize}
\item \hspace*{2mm}\emph{Step 1:} Put the OnionPhone folder on the hard disk, removable media or TrueCrypt container (preferred).

\item \hspace*{2mm}\emph{Step 2:} Edit the TorChat configuration file  $/torchat/bin/Tor/torrc.txt$:
immediately following the line:
\begin{verbatim}
HiddenServicePort 11009 127.0.0.1:11009
\end{verbatim}
add the a new line:
\begin{verbatim}
HiddenServicePort 17447 127.0.0.1:17447
\end{verbatim}
then run the TorChat.

\item \hspace*{2mm}\emph{Step 3:}  Right click on $myself$  icon of TorChat contacts list and copy ID to clipboard. Edit the OnionPhone configuration file $conf.txt$: specify our Onion address using copied ID, for example:
\begin{verbatim}
Our_onion=gegelcy5fw7dsnsn
\end{verbatim}
Once runs the OnionPhone now is ready to receive incoming and make outgoing calls.
\end{itemize}
\subsection{Usage}
\begin{itemize}
\item To accept an incoming call press $Enter$.
\item To make an outgoing call as a guest to guest (without using of personal public keys) type command:  \textbf{-Oremote\_onion\_address} and press $Enter$ then wait 10-30 sec for connecting over Tor.
\item To continuous enabling / disabling of the voice transmission use the $Enter$. Hold down / release the $Tab$  for Push-to-Talk mode.
\item To apply the voice codec from 1 to 18 use the command  \textbf{-Ccodec\_number}. Smaller numbers correspond to low bitrate codec, great - high quality. Numbers from 16 to 18 correspond to the variable bit rate codec.
\item To enable security vocoder use the command \textbf{-Qmode}, where $mode=3$ correspond  ``breathy"  voice (preferred), modes 6-255 correspond  ``robot" etc). For deactivation of vocoder use the command \textbf{-Q-3}.
\item To use the chat feature  type a message and send it by pressing $Enter$.
\item To switch to direct UDP connection use the command \textbf{-S} (both parties must do this).
\item To return into Tor from direct UDP connection use the command  \textbf{-O}.
\item To end the call use the command \textbf{-H}.
\item To exit the OnioPhone use the command \textbf{-X} or click \textbf{Esc} twice for emergence exiting.
\end{itemize}
\section{Keyboard interface}
The keyboard is used to controlling the voice transmission, typing the commands and navigation in menu. Control keys are:
\begin{itemize}
\item \textbf{Back} removes the last typed character.

\item \textbf{Del} clears the typed string.

\item \textbf{Tab} performed voice transmission while key hold down (Push-to-Talk mode).

\item \textbf{Sift+Tab} (Linux) or \textbf{Ctrl+Tab} (Windows) activates the voice detector.

\item \textbf{Up}, \textbf{Down}  arrows used to navigate between menu units.

\item \textbf{Left}, \textbf{Right} arrows used for menus item selection.

\item \textbf{Enter}:

\begin{itemize}
\item answers while incoming call is waits for accepting;
\item if command line is empty enables / disables continuous voice transmission;
\item if the first char of command string is ``\textbf{-}" (command was typed)  process the command;
\item otherwise sends typed chat message.
\end{itemize}

\item \textbf{Esc}:
\begin{itemize}
\item rejects while incoming call is waits for accepting;
\item if clicks twice emergency exits the program clearing the memory.
\end{itemize}
\end{itemize}
\section{Menu}
Console menu allows to quickly outputs commands are most often performed by the user. The menu consists of 9 sections. Navigation between sections is performed using the $Up$ and $Down$ keys. Selecting an item in the section is done using $Left$ and $Right$ keys. The selected item applied using the $Enter$ key. Menu's sections are:
\begin{enumerate}
\item \textbf{Controls} - termination of the call, exit the application, obtaining information about the current state, measuring the latency of connection, controls of voice processing (loopback).

\item \textbf{Low bitrates codec} - the selection of codec, obtaining information information about the current codec.

\item \textbf{High-quality codec} - the selection of codec, obtaining information about the codec used by the other party.

\item \textbf{Voice processing} - enabling / disabling the automatic gain control, noise suppression and vocoder.

\item \textbf{Jitter compensation} - specification of  fixed size buffer or auto-adjust depending on the parameters of the communication channel.

\item \textbf{Voice detector} - activation the transmission mode using mike level or voice analyzer.

\item \textbf{Signal levels} - activation of the three-tone alarm received from the opposite side after voice message and noise substitution of lost fragments.

\item \textbf{Switching} - establish UDP direct connection using selected STUN-server for NAT traversal.

\item \textbf{Contacts} - can view the address book and select the contact for outgoing call.

\item \textbf{Others} - setup the backup counter Onion connection, reset application etc.
\end{enumerate}
\section{Commands}
The command should start with a ``\textbf{-}" symbol followed capital letter followed optional parameter. No spaces can be between letter and parameter. Some commands can contain several subcommands separated by spaces. After typing the commands press $Enter$ for processing. While typing use $Back$ for correction of typed characters and $Del$ for canceling and cleaning the command string.
\subsection{Address book management}

\textbf{-V[filter] }displays all the entries of the address book containing the substring $filter$. If the parameter to be omitted displays all uncommented entries.

\noindent \textbf{-E[name]} outputs the command for call to the subscriber $name$ defined in the address book. If parameter to be omitted outputs the most recently used call command (redialling).

\subsection{Application and Connection management}
\textbf{-H} graceful terminated the call.

\noindent  \textbf{-X} graceful shutdown the program.
\subsection{Sound management}

\textbf{-C[codec] }sets the codec for outgoing voice stream. Parameter $codec$ is codec number from 1 to 18. If the parameter is ``?", the commands shows the number of the current codec. If omitted, the default codec applies.

\noindent \textbf{-J[jitter]} sets a fixed size of buffer to compensate transport jitter. Parameter $jitter$ is average buffering delay in milliseconds. If you use the parameter=``?" the current value displays.  If ``-1" the minimal buffering applies (lower latency but bad resistance to jitter).  If the parameter is omitted the automatic buffering adaptation to the communication channel will use (default).

\noindent \textbf{-Q[voice]} sets the style of  voice preprocessing. Parameter $voice$ is:  ``0"-no automatic microphone gain control (AGC) and no noise suppressing (NPP), ``1"-applies AGC, ``2"-applies NPP, ``3" - applies a vocoder in ``breathy" mode, ``4" - in ``high" mode, ``5"- in ``deep" mode,  ``6"-``255" -  in ``robot" mode with different pitch, ``-1"-disables AGC, ``-2"-disables NPP, ``-3"  - disables the  vocoder (unchanged voice will be sent), ``?" - show current state of voice processing.

\subsection{Key sharing}

\textbf{-K[name]} sent specified public key $name$ from the our phone book to remote side. This command is only possible if the connection is active. If parameter omitted sent own public key specified in the configuration file. The keys are receives and automatically adds to the address book with option \textbf{-L}  without parameter (untrusted). Later user can view the address book and check out all the untrusted keys (check the PGP signature into key file) and edit the trust level to that key.
\subsection{Authentication}


\textbf{-Yaccess} specifies password for private key correspondent for $our\_name$ used for processing all incoming and default outgoing calls. If private key is in encrypted format then this option must be applied immediately after OnionPhone runs otherwise you can not accept incoming calls.  If parameter is space applied  key access will be cleared. If parameter omitted applied key access will be checked for validity for specified secret key correspond our default name.

The alternative way to introduce access is specify password as $Our\_secret\_access$ parameter in configuration file. In this case it is recommended to store OnionPhone folder  in a protected place (TrueCrypt container etc.). The access also can be temporarily applied to the current outgoing call using \textbf{-Yaccess} into the  call extended command \textbf{-N} (see below).

\noindent \textbf{-P[password]} applies a common password for authentication and initiates it. Passwords can contain from 3 to 31 characters and must be the same for both parties except for the last two characters: one of the participants swaps their places. The password can be used both during the established connection and when dialing (see the extended command for calls ), as well as defined for each user in the address book (in this case it will be applied automatically when an incoming call from that person occurred). If parameter omitted applied password will be cleared.

In the case of manually entering a password (preferably for safety) provides the ability to hidden alerts the other side  being  under duress: user can changes the last character of the password on any other. In this case the authentication on his side will look as usual but on the opposite side participant receive a warning. An attacker can not verify in any way the correctness of the last letter of the password  even when analyzing dumps of previous sessions, controlling current session, as well as with the active ``fishing"  by creating test sessions to both parties, both before and after.

\noindent \textbf{-W[interval]} sets the interval to reconnect the slower of the two Onion connections (to reduce the overall latency of the connection). If parameter omitted the default value is used. If the parameter is �0�, the counter connection is not uses for data exchanging  and breaks immediately after verifying the Onion address. If  ``-1"  the counter connection is not established (also Onion address verification is not possible). If the parameter is too large  (e.g., ``1000000000"), the counter connection is used but never reconnected.

\noindent \textbf{-WI} command initiates verification originator's Onion address  by re-creating reverse connection to the declared hidden service and exchanging of verifiers (like TorChat authentication procedure). Command can be used only if the Tor connection to Onion address used and originator allowed verification. If the originator specified Onion address in configuration file and under the above conditions verification is performed automatically after the key agreement. Manually execution of this command is useful if the process was frozen.

\noindent \textbf{-WR} command gracefully restored doubling process by sending the other side invite with a request to restart  their outgoing connection.

\subsection{Switch to direct connection}

\textbf{-S[stun] }Initiates switching from Onion connection to direct UDP connection using specifies STUN-server for the NAT traversal. If parameter omitted the default STUN specified in configuration file will be used. If  ``0",  NAT traversal attempts are not performed (in this case the direct connection is only possible if both parties are on the same subnet or if at least one of the parties has the ``real" IP-address on the interface of computer running OnionPhone.

Command can be executed only if the Onion connection is already established. Both parties must executes this command to start process. If both parties are behind the full-cone NATs, or both are in a common subnet with the local IP (for example, home internet provider), and then each behind own router, the NAT traversal can fail. In this case users will be able to continue communication over Tor.
After a successful switch to direct UDP-connection the Onion connection is still available but not actually used for communication . At any time any user can stop the direct connection and return to the use of the Onion connection. To do this at least one of the participants must execute command \textbf{-O} with no parameter.

\subsection{Extended call command}

\textbf{-N[name] destination\_subcommad [options\_subcommands]} initiates a connection with the subscriber $name$ (the subscriber�s public key must be added  to address book). If the name is omitted the default $guest$ will be used. Public and private keys for the guest included in the OnionPhone by default and are available for each user (and also for the attacker). The authentication using $guest$ key is not performed  and there is a danger MitM-attack in the case of installation of direct connection. In the case of  connection over Tor the key agreement is going on inside Tor so MitM like impossible (while you trust Tor).

Parameter $destination\_commad$ specifies the type of transport (UDP, TCP, Onion) and the address of the recipient (IP or hostname and optionally port number separated by ``:"), for example:
\begin{verbatim}
-U178.95.218.29:17447
\end{verbatim}
\begin{verbatim}
-Talice.dyndns.org
\end{verbatim}
\begin{verbatim}
-Or4kxspnzpnsel4fu
\end{verbatim}
Connection type \textbf{-U} initiates direct UDP-connection, \textbf{-T} for direct TCP-connection, \textbf{-O} for TCP-connection over Tor. Address can be specified as a string IP, domain name or Onion address  with or without the suffix $.onion$. The port can be specified after the address separated by a colon. If no port is specified the default port  17447 will be used.

Further separates by spaces can follow subcommands  \textbf{-P[password]} (see above) and \textbf{-I[our\_name]}:
$our\_name$ parameter specifies the name of the originator appears to the recipient. If this option is completely omitted the proper name is defaults as defined in the configuration file. If \textbf{-I }option is used without a parameter the default name $guest$ will be uses (see above). If parameter is specified will use the specified name as proper name for current session. In this case the initiator should have files of both public and private keys for this name. In addition this public key must be added to the address book. Of course the recipient also must  have a public key for this name and this key must be added in his address book otherwise the connection will not be established.

Option \textbf{-Yaccess} can  follow specifying the password to the secret key $our\_name$. Note this password temporary (only for this call) overwrites  access to default private key used for incoming calls.
If subscribers are connected first with each other they can be assured only in the presence of guest key on both sides. Thus the first connection must be initiated as untrusted $guest \rightarrow guest$:
\begin{verbatim}
-N -I -Oonion_address
\end{verbatim}
Such this connection is still encrypted with the session key but be resistant to MitM just so so you trust Tor. Also the parties can't be sure of the authenticity of each other (except for the possible verification Onion addresses similar to TorChat). Preinstalled untrusted connection can share PGP-signed long-term public keys using \textbf{-K} command (this is denied because keys are public and attacker can't be sure that customers really declare own key: user can also send multiple keys of other various users) . Received keys will be automatically added to the address books and marked as untrusted. After manually checking of the PGP signatures users can set the desired level of trust by editing contacts entries in the address book file. Later user can re-install already trusted connection, for example:
\begin{verbatim}
-Nalice -Or4kxspnzpnsel4fu -Ibob
\end{verbatim}
\section{Key management}
OnionPhone uses public keys to authenticate the subscribers to each other using PGP. Also the keys in this format will be used in other projects in the future (audio chat group, encryption radio and over-GSM-audio channels, etc.).   Console utility $addkey$ useful for key management:

\hspace*{2mm}\emph{Step 1}: To generate a new key pair use:
\begin{verbatim}
./addkey -Gname [-Yaccess] [options]
\end{verbatim}
where $name$ is your identifier, $access$ is password that will be encrypted private key and options will be passed to other participants. If \textbf{-Y}  is omitted the private key will be stored unencrypted. In this case it is recommended to store OnionPhone folder  in a protected place (TrueCrypt container etc.).

The most common use is option \textbf{-Oour\_onion\_address} to present own address. Upon receipt of your key this option  will be automatically copied in address books of other participants. For example, after the command
\begin{verbatim}
./addkey -Galice -Y1234 -Or4kxspnzpnsel4fu
\end{verbatim}
files $alice.sec$ (the private key encrypted  using password $1234$) and $alice$ (the public key) will be created in the folder $keys$.

\hspace*{2mm}\emph{Step 2}: To sign the public key open it as a text and sign their content using PGP. Signature must be added  to the key file. Once the key has been signed further edition  is not allowed (rename still possible).

\hspace*{2mm}\emph{Step 3}: Before using the key you must add it to your address book:
\begin{verbatim}
./addkey -Aalice
\end{verbatim}
To use this key as its own default edit the OnionPhone configuration file $conf.txt$ specifying the key's name: $Our\_name=alice$

\hspace*{2mm}\emph{Step 4}: Now you can send the key to other users. For transmission of the key establish  un-authenticated call $guest \rightarrow guest$ with the other party (see above) and execute
\textbf{-Kalice} (or \textbf{-K} without parameter if this key is assigned as your own default). Key will be automatically added to the remote address book with the lower level of trust. After manual checking of PGP signature in the key file other parties can set the desired trust level  by editing \textbf{-L} parameter to the corresponding entries in the address book file $keys / contacts.txt$, for example:
\begin{verbatim}
[alice] {FNrjEjGmlZtvKXzBQkNIDA ==} #alice -Or4kxspnzpnsel4fu -L1
\end{verbatim}
You can also pass the key in any other way (for example, by email). The recipient verifies PGP signature in the key file and determines the necessary trust level, then add the key to the address book indicating the trust level as \textbf{-Llevel}, for example:
\begin{verbatim}
./addkey -Aalice -L1
\end{verbatim}

\newpage
\section{Cryptography design}
Normally OnionPhone establishes a connection over Tor network. Despite the use of  Tor as a secure transport OnionPhone provides own layes of a p2p encryption and multifactors authentication. This allows  to use OnionPhone for direct connections out of  Tor.
\subsection{Cryptograhic Primitives}
\begin{itemize}
\item Asymmetric cryptography is Diffie - Hellmann on EC25519  \cite{Bernstein06} used curve25519\_donna C-code\cite{Langley11} ;
\item Symmetric cryptography is stream encryption used Keccak Sponge Dulpexing library\cite{Gegel13} ;
\item SPRNG is Keccak Sponge Duplexing RNG \cite{BDPA10, BDPA111} with Havege reseeding.
\item Hashing, MAC, KDF are Keccak in 576/1600 mode\cite{BDPA11} ;
\end{itemize}
We don't use special MAC and KDF functions  i.e. Keccak is RO-PRF and can be directly uses  for this purpose.
\subsection{Symmetric encryption}
Uses stream cipher based on Keccak.  Both parties maintain synchronized counters $ctr$ separately for sent and received packets. TCP transport is lossless and provides synchronization of counters by himself. For UDP each packet's header contains last 7 bit of sender's outgoing counter value for synchronization incoming counter on receiver�s side.
For each packet Keccak Sponge initiates and absorbs 128-bits session key $s$, 32-bits counter $ctr$ and 8-bits flag $O$ is $0$ for an originator and $1$ for an acceptor of the call,  then squeezes gamma $g=H(s | ctr | O)$.

Gamma XOR-ed with plainest $p$ (packet's bytes except first byte as a header $h$) provides ciphertext: $c=g \oplus p$.
After packet was encrypted 32-bits MAC computed: $mac=H(s | ctr | O | h | c)$,
After this counter increments by 1. There is no way to return counter back.

Encrypted packet is $h, p, mac$. Decryption is similar to encryption.

\subsection{Key formats}
OnionPhone uses ECDH25519 256 bits private keys and 256 bits Montgomery public keys providing about 128 bit security level.

\textbf{Public key }is a text file contains: \textbf{Info}, \textbf{Key}, \textbf{Sig}(Info, Key), where:
\begin{itemize}
\item \textbf{Info} string with first symbol ``\#", name and options.
Name is up to 31 characters string and option is symbol ``-",  followed by option type (capital letter),  followed by  string of parameters. Options separated by spaces. All options passes to address book for further processing by application.

Some other strings are optional comments etc.
\item \textbf{Key} string is base64 representation of ECDH25519 public key (in figure brackets).
\item \textbf{Sig} is a PGP signature as a part of key file: info-string and key string must be signed by PGP.
\end{itemize}
Key stamp ($ID$) is a first 128 bits of Keccak hash of  public key file. After public key was generated  and signed anyone can't modify it  but can optionally rename the file. Owner must add own public key to address book before using. All other parties also must add this key before using.

\emph{Note:} public key is not self-signed. An attacker could re-sign someone else's key submit it as their own thereby trying to mount an UKS attack. This can be prevented including $ID$ in the authentication process. Implemented initial key exchange protocol completely solved this problem.

\textbf{Private key} is a binary file contains 256 bits ECDH25519 secret. Private key can be saved unencrypted (32 bytes) or encrypted by password (64 bytes). Both formats are acceptable but encrypted format requires the use of command \textbf{-Yaccess} when working with the OnionPhone.
For encryption of private key uses symmetric key derived from the password: initiates a Keccak Sponge, then once absorbs  the $salt$ string ``\$OnionPhone\_salt", then absorbs $password$ string 32768 times in a row, and then squeezes a 16-bytes key:
$s=H(salt | password | password | \ldots | password)$.

To perform encryption Keccak Sponge initiates and absorbs key $s$ and randomly generated 16-bytes $nonce$, then squeezes a 32-bytes gamma: $g=H(s | nonce)$.
Gamma XOR-ed  with private key�s body $x$ to produce encrypted data: $c=g \oplus x$.
After packet was encrypted  16-bytes MAC computes: $mac=H(c | s | nonce)$.
Encrypted private key is $c, nonce, mac$.

\textbf{Key pair ``guest"}  is a private key as a 32 bytes binary representation of string ``Guest secret http://torfone.org" and corresponding public key. This keys included by default and can be used for performing primary unauthenticated connection with an unknown receiver.

\subsection{Deniability authenticated Key agreement}
The core of implemented protocol is \texttt{SKEME}. The shared secret is derived only using the Diffie - Hellman exchange. Public parameters are signed by another key outputted by another protocol performed in parallel. The original \texttt{SKEME} \cite{Krawczyk96} uses the \texttt{Abadi} protocol \cite{AF03} as auxiliary. For this purpose we use a slightly modified \texttt{KEA+} protocol\cite{LM06} . This modification is known as the \texttt{TripleDH} and is used, for example, as a initial key exchange (IKE) in the \texttt{Axolotl} protocol\cite{TM13} .

OnionPhone's IKE provides key agreement (produces fresh session key for each run achieving Prefect Forward Secrecy) and fully deniable authentication using long-term ECDH public keys signed by PGP.  If the parties can deny having exchanged a key with the other party then the rest of the communication can also be denied.

Proposed IKE intuitively view as deniable in both Deng at all \cite{DLZ01} (full deniability) and Raimondo and Gennaro (forward deniability) \cite{RG09} scenario even with a cooperating judge: while a dishonest party cooperate with judge before protocol running (but not during IKE) it still will not be able to convince the judge after IKE will be completed that the other party is really participated in the protocol. The initiator's view of the protocol can be simulated by an simulator that does not know the secret key of the acceptor and vice versa. The value of the session key also can be part of the output of the simulation so all subsequent session messages can also be simulated. Furthermore provides the ability of revealing of auxiliary MAC key after completing the IKE. This is a not a standard cryptographic way that's why this is optionally.

The IKE looks like resistance to UKS and wPFS attacks. Note that full deniability property conflicts with KCI resistance. Therefore after the leak of  user's private key the attacker can impersonate anybody to him. In our case the attack can be mounted only when the victim never been in contact with the impersonated party and does not have his real public key yet. Fortunately the IKE is sensitive only to the leak of ECDH private key but not to the leak of  PGP private key used to sign.

 \medskip
\textbf{Protocol:} let $p$,$q$ are two large primes satisfying $q|(p-1)$, $G_1$ is an additive group of order $q$ on an elliptic curve 25519.  Public key $T = g^t$, where $t$ $\in$ $Z_q$ is the private key and $g=9$ is the base point of $G_1$. $H$ is a one-way hash function with RO-PRF property (SHA3 Keccak Sponge).

When Alice wants to exchange session key $s$ with Bob, they cooperatively perform the following
steps. Here, Alice�s and Bob�s identities, public and private keys are $id_A$, $A$, $a$ and $id_B$, $B$, $b$ respectively. Both participants randomly picks nonce $N_A$ and $N_B$, two ephemeral secret keys $x$, $v$  and $y$, $w$ and computes public keys $X$, $V$ and $Y$, $W$  respectively.

The first Alice and then Bob computes and send each other their  randomly-looking hidden identifiers $D_A=H(id_A | B^v )$  and $ D_B=H(id_B | V^w )$, session public keys and nonce. Note that Alice the first commits her $X$, $N_A$ values and opens they only after receiving $Y$, $N_B$ from Bob.

Then participants shares the common auxiliary key $K=H(W^a | B^v | W^v)=H(A^w | V^b | V^w)$ and computes authenticators signing the public session parameters:  $M_A=H(K | V | W | X | Y | N_A | N_B)$ and $M_B=H(K | W | V | Y | X | N_B | N_A)$. After an exchange of authenticators and check them both participants can be assured of the identity of each other. Finally  participants shares the session key $s=H(X^y) \oplus N_A \oplus N_B = H(Y^x) \oplus N_A \oplus N_B$ and optionally reveals $K$.

 \pagebreak
Thus, to establish a session key parties $A$ and $B$ shall exchange the following messages:
\begin{enumerate}
\bigskip
\item $A$ (originator):

 $ D_A=H(id_A | B^v ), V=g^v, X=g^x$, where $v$ $\in$ $Z_q$, $x$ $\in$ $Z_q$ at random

$C=H(X|N_A)$, where $N_A$ at random

$A \rightarrow B: $\hspace*{10mm}$ D_A,V, C$
\bigskip
\item $B$:

searches contact in $id$-list matched $D_A$

$D_B=H(id_B | V^w), W=g^w, Y=g^y$, where $w$ $\in$ $Z_q$, $y$ $\in$ $Z_q$ at random

$N_B$ at random

$A \rightarrow B: $\hspace*{10mm}$D_B, W, Y, N_B$
\bigskip
\item $A$:

checks $D_B$ valid?

$K=H(W^a | B^v | W^v)$

$M_A=H(K| V | W | X | Y | N_A | N_B)$

$s=H(X^y) \oplus N_A \oplus N_B$

$A \rightarrow B: $\hspace*{10mm}$X, N_A, M_A$
\bigskip
\item $B$:

checks $C$ and $M_A$ valid?

$K=H(A^w | V^b | V^w)$

$M_B=H(K | W | V | Y | X | N_B | N_A)$

$s=H(Y^x) \oplus N_A \oplus N_B$

$B \rightarrow A: $\hspace*{10mm}$M_B$

\end{enumerate}
\bigskip
$\hspace*{5mm}$A$: $\hspace*{10mm}checks $M_B$ valid?

\begin{center}
\bigskip
$A, B: $\hspace*{2mm}shares secret $s$

$A \rightarrow B: $\hspace*{2mm}$K$ (optionally reveals auxiliary key)

\end{center}
\newpage
\subsection{Voice authentication}
Voice (biometric) authentication is similar to SAS authentication in ZRTP \cite{Zimmermann10} and is useful to prevention the MitM attack if the participants know each other by voice. Participants must read each other a list $L$ of words based at the shared secret $s$. Since uses short authenticators (only 32 bits for the list of four words of the 256 available) commitment is needed to prevent the dishonest party influence of the shared secret to obtain the same SAS during the MitM attack:
\begin{enumerate}
\item $A \rightarrow B: $\hspace*{10mm} $C=H(N_1)$, where $N_1$ at random
\item $B \rightarrow A: $\hspace*{10mm} $N_2$ at random
\item $A \rightarrow B: $\hspace*{10mm} $N_1$, $L=H(N_1 | N_2 | s)$
\item $B \rightarrow A: $\hspace*{10mm} $L=H(N_2 | N_1 | s)$
\end{enumerate}

\subsection{Optional authentication using pre-shared password}
This authentication can be performed for the detection of MitM attack after the key agreed and the shared secret derived. Password must be identical in both parties except last two chars: one of the parties swaps them with each other. Lets  Alice's password is $pass|p|q$ and Bob's password is $pass|q|p$. The participant under pressing must change last char of own password on any other. Authentication will look like correct from his side but other party will be notified. Each party can initiate authentication independently. For example, Alice is the originator of call and initiates authentication using pre-shared password:
\begin{enumerate}
\item $A \rightarrow B: $\hspace*{10mm} $U1_A=H( O | s | pass )$

where $s$ is shared secret, $O$ is byte $0$ for originator and $1$  for acceptor of call;

\item $B: $ Checks $U1_A$, rejects invalid;

$B \rightarrow A: $\hspace*{10mm} $U1_B=H( O+2 | s | pass )$, $U2_B=H( O+2 | s | pass | p )$

\item $A: $ Checks $U1_B$, rejects invalid; checks $U2_B$, warns invalid;

$A \rightarrow B: $\hspace*{10mm} $U2_A=H( O | s | pass | q )$

\item $B: $ Checks $U2_A$, warns invalid;
\end{enumerate}
\newpage
\subsection{Optional Originator's onion address verification}
After originator was connects to acceptor's Hidden Service (HS) he is sure of acceptor's Onion address (while he trust Tor). But acceptor can't know initiator's Onion address: initiator must present it to acceptor. Acceptor can trust it only after it's verification.

Lets Alice is originator and Bob accepts the connection to its HS, $s$ is shared secret, $O$ is byte $0$ for originator and $1$  for acceptor of the call:
\begin{enumerate}
\item $A: $ Connects to Bob's HS using their Onion address.

$B \leftrightarrow A: $ Parties agree secret $s$ as describe above using channel 1.
\item $B: $ Requests for Alice's  Onion address using channel 1:

$B \rightarrow A: $\hspace*{10mm} $?$
\item $A: $ Sends own Onion address to Bob using channel 1 (encrypted):

$A \rightarrow B: $\hspace*{10mm} $alice.onion$
\item $B: $ Connects to HS specifies by received Onion address, sends using channel 2:

$B \Rightarrow A: $\hspace*{10mm} $I_R=H( O+4 | s )$
\item $A: $ Checks $I_R$, rejects invalid; sends using channel 2:

$A \Rightarrow B: $\hspace*{10mm} $I_S=H( O+4 | s )$
\item $B: $ Checks $I_S$, warns invalid.

Now Bob can be sure that remote Onion address matches submitted by Alice.
\end{enumerate}

For prevents unwanted verification  originator can't specified our Onion address in configuration file. So, the recipient does not know the Onion address of the originator and the originator remains completely anonymous.

\newpage
\begin{thebibliography}{4}

\renewcommand{\baselinestretch}{1.0}
  \renewcommand{\arraystretch}{.9}
  \normalsize \small \parskip 1mm

 \bibitem{Gegel12} Van Gegel. (2012).  TOR  Fone  -  p2p  secure  and  anonymous  VoIP  tool.
 \href{http://torfone.org/}{(Web link)}
 %Avaliable: \verb|http://torfone.org/|

  \bibitem{Meredith13}   Leslie Meredith. 3 Privacy Tools to Stop Gov. Phone Snooping. TechNewsDaily [Jun 7, 2013] 1:(1)[4 screen].
  \href{http://news.discovery.com/tech/gear-and-gadgets/3-privacy-tools-stop-gov-snooping-130607.htm)}{(Web link)}

 %Avaliable: \verb|http://news.discovery.com/tech/gear-and-gadgets/3-privacy-tools-stop-gov-snooping-130607.htm|

  \bibitem{Rizal14} Maimun Rizal. A Study of VoIP Performance in Anonymous Network - The Onion Routing (Tor) [dissertation] Georg-August Univ.; Gottingen, 2014. \href{https://ediss.uni-goettingen.de/bitstream/handle/11858/00-1735-0000-0022-5EF2-D/Dissertation%20Maimun__Rizal_final.pdf?sequence=1}{(Full text)}

 %Avaliable: \verb|https://ediss.uni-goettingen.de/bitstream/handle/11858/00-1735-0000-0022-5EF2-D/Dissertation%20Maimun__Rizal_final.pdf?sequence=1|

 \bibitem{KFK14} Georgios Karopoulos, Alexandros Fakis, Georgios Kambourakis. Complete SIP message obfuscation: PrivaSIP over Tor (University of the Aegean). The Ninth International Workshop on Frontiers in Availability, Reliability and Security (FARES); University of Fribourg, Switzerland; Sep. 8th - 12th, 2014.
     \href{http://www.icsd.aegean.gr/publication_files/conference/314865552.pdf}{(Full text)}

 %Avaliable: \verb|http://www.icsd.aegean.gr/publication_files/conference/314865552.pdf|

\bibitem{Kambourakis14} Georgios Kambourakis. Anonymity and closely related terms in the Cyberspace: An analysis by example. Journal of information security and applications XXX (2014), 1:16.
    \href{http://libgen.org/scimag/get.php?doi=10.1016/j.jisa.2014.04.001}{(Full text)}
%Avaliable: \verb|http://phdtree.org/pdf/39023717-anonymity-and-closely-related-terms-in-the-cyberspace-an-analysis-by-example/|

  \bibitem{SLL14} Keen Sung, Brian Neil Levine,  Marc Liberatore. Location Privacy without Carrier Cooperation
(Univ. of Massachusetts Amherst). IEEE Symposium on Security and Privacy (SP) San Jose, CA
Mobile Security Technologies (MoST); May 17, 2014.
\href{http://mostconf.org/2014/papers/s1p3.pdf}{(Full text)}

%Avaliable: \verb|http://mostconf.org/2014/papers/s1p3.pdf|

 \bibitem{Zimmermann99} Phil Zimmermann. (1999) PGPfone - Pretty Good Privacy Phone.
 \href{http://www.pgpi.org/products/pgpfone/}{(Web link)}
 %Available: \verb|http://www.pgpi.org/products/pgpfone/|

  \bibitem{Gegel14} Van Gegel  (2014). OnionPhone  -  VOIP add-on for TorChat.
  \href{http://torfone.org/onionphone/}{(Web link)}

%Avaliable: \verb|http://torfone.org/onionphone/|

    \bibitem{Kreuss12} Bernd Kreuss (2012) TorChat: Decentralized anonymous instant messenger on top of Tor Hidden Services.
     \href{https://github.com/prof7bit/TorChat}{(Web link)}
%Available: \verb|https://github.com/prof7bit/TorChat|

   \bibitem{RGK06} Mario Di Raimondo, Rosario Gennaro, Hugo Krawczyk. Deniable Authentication and Key Exchange. Proceedings of the 13th ACM conference on Computer and communications security; 2006; 400:409.
       \href{https://www.dmi.unict.it/diraimondo/web/wp-content/uploads/papers/deniability-ake.pdf}{(Full text)}
%Available: \verb|https://www.dmi.unict.it/diraimondo/web/wp-content/uploads/papers/deniability-ake.pdf|

   \bibitem{BZ11} William Andrew Burnson, Wenjia Zhou. Real-Time Voice Conversion: A Multirate 8kHz LPC Vocoder. [Thesis Research] Univ. of Illinois, 2011.
       \href{http://williamandrewburnson.com/media/RealTimeVoiceConversion.pdf}{(Full text)}

%Available: \verb|http://williamandrewburnson.com/media/RealTimeVoiceConversion.pdf|

     \bibitem{Gegel141} Van Gegel (2013) OnionPhone  -  VOIP add-on for TorChat.
     \href{https://github.com/gegel/onionphone}{(Web link)}
%Available: \verb|https://github.com/gegel/onionphone|
 \bibitem{Bernstein06} Daniel J. Bernstein Curve25519: new Diffie-Hellman speed records. In Public Key Cryptography (PKC), Springer-Verlag LNCS 3958, 2006.
     \href{http://cr.yp.to/ecdh/curve25519-20060209.pdf}{(Full text)}

%Avaliable: \verb|http://cr.yp.to/ecdh/curve25519-20060209.pdf|

    \bibitem{Langley11} Adam Langley (2011) Implementations of a fast Elliptic-curve Diffie-Hellman primitive.
    \href{http://cr.yp.to/ecdh/curve25519-20060209.pdf}{(Full text)}

%Available: \verb|http://cr.yp.to/ecdh/curve25519-20060209.pdf|

    \bibitem{Gegel13} Van Gegel (2013) Portable C implementation of universal Sponge construction based on compact Keccak source code.
        \href{https://github.com/gegel/sponge}{(Web link)}

%Available: \verb|https://github.com/gegel/sponge|

     \bibitem{BDPA10} Guido Bertoni, Joan Daemen, Michael Peeters,  Gilles Van Assche. Sponge-based pseudo-random number generators. Cryptographic Hardware and Embedded Systems, CHES 2010. 12th International Workshop, Santa Barbara, USA, August 17-20, 2010; 33:47.
         \href{www.iacr.org/archive/ches2010/62250031/62250031.pdf}{(Full text)}

%Avaliable: \verb|http://www.iacr.org/archive/ches2010/62250031/62250031.pdf|

     \bibitem{BDPA111} Guido Bertoni, Joan Daemen, Michael Peeters,  Gilles Van Assche. Duplexing the sponge: single-pass authenticated encryption and other applications. Selected Areas in Cryptography. 18th International Workshop, SAC 2011, Toronto, ON, Canada, August 11-12, 2011, Revised Selected Papers, 2011; 320:337.
         \href{http://eprint.iacr.org/2011/499.pdf}{(Full text)}

%Available: \verb|http://eprint.iacr.org/2011/499.pdf|

    \bibitem{BDPA11} Guido Bertoni, Joan Daemen, Michael Peeters,  Gilles Van Assche. The Keccak reference. Version 3.0.; January 14, 2011.

        \href{http://eprint.iacr.org/2011/499.pdfhttp://keccak.noekeon.org/Keccak-reference-3.0.pdf}{(Full text)}

%Avaliable: \verb|http://keccak.noekeon.org/Keccak-reference-3.0.pdf|

 \bibitem{Krawczyk96} Hugo Krawczyk. SKEME: A Versatile Secure Key Exchange Mechanism for Internet.  Network and Distributed System Security; San Diego, CA, 1996; 114:127.
     \href{http://www.di.unisa.it/~ads/corso-security/www/CORSO-9900/oracle/skeme.pdf}{(Full text)}

%Available:  \verb|http://www.di.unisa.it/~ads/corso-security/www/CORSO-9900/oracle/skeme.pdf|

\bibitem{AF03} Martin Abadi, Cedric Fournet. Private Authentication  In Software Security � Theories and Systems. Mext-NSF-JSPS International Symposium (ISSS�02); 2003; 317:338.
    \href{http://users.soe.ucsc.edu/~abadi/Papers/tcs-private-authentication.pdf}{(Full text)}

%Available: \verb| http://users.soe.ucsc.edu/~abadi/Papers/tcs-private-authentication.pdf|

\bibitem{LM06} Kristin Lauter, Anton Mityagin. Security Analysis of KEA Authenticated Key Exchange Protocol. PKC 2006, volume 3958 of LNCS; 2006; 378:394.
    \href{http://research.microsoft.com/en-us/um/people/klauter/security_of_kea_ake_protocol.pdf}{(Full text)}

%Available: \verb|http://research.microsoft.com/en-us/um/people/klauter/security_of_kea_ake_protocol.pdf|

      \bibitem{TM13} Trevor Perrin, Moxie Marlinspike (2013). Advanced cryptographic ratcheting.
      \href{https://www.whispersystems.org/blog/advanced-ratcheting/}{(Web link)}

%Available: \verb|https://www.whispersystems.org/blog/advanced-ratcheting/|

         \bibitem{DLZ01} X.Deng, C.H.Lee , H.Zhu.  Deniable authentication protocols. IEE Proc.Comput.Digital Techniques, 2001, 148(2):101-104.
             \href{http://digital-library.theiet.org/content/journals/10.1049/ip-cdt_20010207}{(Web link)}

       \bibitem{RG09} Mario Di Raimondo, Rosario Gennaro. New Approaches for Deniable Authentication. Journal of Cryptology,
2009, 22(4):572-615.
\href{https://eprint.iacr.org/2005/046.pdf}{(Full text)}

%Avaliable: \verb|https://eprint.iacr.org/2005/046.pdf|

          \bibitem{Zimmermann10} Phil Zimmermann. ZRTP: Media Path Key Agreement for Unicast Secure RTP. Network Working Group Internet-Draft; June 17, 2010; 1:31.
              \href{https://tools.ietf.org/html/draft-zimmermann-avt-zrtp-22#section-4.5.1}{(Web link)}

%Avaliable: \verb|https://tools.ietf.org/html/draft-zimmermann-avt-zrtp-22#section-4.5.1|

\end{thebibliography}

\end{document}
